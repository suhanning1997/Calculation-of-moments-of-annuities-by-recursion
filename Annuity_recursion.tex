\documentclass[]{article}\usepackage[]{graphicx}\usepackage[]{color}
% maxwidth is the original width if it is less than linewidth
% otherwise use linewidth (to make sure the graphics do not exceed the margin)
\makeatletter
\def\maxwidth{ %
  \ifdim\Gin@nat@width>\linewidth
    \linewidth
  \else
    \Gin@nat@width
  \fi
}
\makeatother

\definecolor{fgcolor}{rgb}{0.345, 0.345, 0.345}
\newcommand{\hlnum}[1]{\textcolor[rgb]{0.686,0.059,0.569}{#1}}%
\newcommand{\hlstr}[1]{\textcolor[rgb]{0.192,0.494,0.8}{#1}}%
\newcommand{\hlcom}[1]{\textcolor[rgb]{0.678,0.584,0.686}{\textit{#1}}}%
\newcommand{\hlopt}[1]{\textcolor[rgb]{0,0,0}{#1}}%
\newcommand{\hlstd}[1]{\textcolor[rgb]{0.345,0.345,0.345}{#1}}%
\newcommand{\hlkwa}[1]{\textcolor[rgb]{0.161,0.373,0.58}{\textbf{#1}}}%
\newcommand{\hlkwb}[1]{\textcolor[rgb]{0.69,0.353,0.396}{#1}}%
\newcommand{\hlkwc}[1]{\textcolor[rgb]{0.333,0.667,0.333}{#1}}%
\newcommand{\hlkwd}[1]{\textcolor[rgb]{0.737,0.353,0.396}{\textbf{#1}}}%
\let\hlipl\hlkwb

\usepackage{framed}
\makeatletter
\newenvironment{kframe}{%
 \def\at@end@of@kframe{}%
 \ifinner\ifhmode%
  \def\at@end@of@kframe{\end{minipage}}%
  \begin{minipage}{\columnwidth}%
 \fi\fi%
 \def\FrameCommand##1{\hskip\@totalleftmargin \hskip-\fboxsep
 \colorbox{shadecolor}{##1}\hskip-\fboxsep
     % There is no \\@totalrightmargin, so:
     \hskip-\linewidth \hskip-\@totalleftmargin \hskip\columnwidth}%
 \MakeFramed {\advance\hsize-\width
   \@totalleftmargin\z@ \linewidth\hsize
   \@setminipage}}%
 {\par\unskip\endMakeFramed%
 \at@end@of@kframe}
\makeatother

\definecolor{shadecolor}{rgb}{.97, .97, .97}
\definecolor{messagecolor}{rgb}{0, 0, 0}
\definecolor{warningcolor}{rgb}{1, 0, 1}
\definecolor{errorcolor}{rgb}{1, 0, 0}
\newenvironment{knitrout}{}{} % an empty environment to be redefined in TeX

\usepackage{alltt}

\usepackage[margin=0.75in, paperwidth=8.27in, paperheight=11.69in]{geometry}

\usepackage{parskip}

\usepackage{amsfonts}

\usepackage{amssymb}

\usepackage{amsmath}

\usepackage{amsthm}

\usepackage{bm}

\usepackage{commath}
\IfFileExists{upquote.sty}{\usepackage{upquote}}{}
\begin{document}

\title{\textbf {Unit 2 Slide 121 Example Solution Using R}}
\author{
		Hanning Su \\
        Student ID: 855767\\
}
\maketitle

\section*{R code}

We first calculate $\mu_r$ by the formula:
$$\mu_r=\mathbb{E}\left[\left(1+i_k\right)^{-r}\right]=\frac{0.15}{1.05^r}+\frac{0.60}{1.06^r}+\frac{0.25}{1.07^r}$$

\begin{knitrout}
\definecolor{shadecolor}{rgb}{0.969, 0.969, 0.969}\color{fgcolor}\begin{kframe}
\begin{alltt}
\hlstd{mu_1} \hlkwb{<-} \hlstd{(}\hlnum{1}\hlopt{/}\hlnum{1.05}\hlstd{)}\hlopt{*}\hlnum{0.15} \hlopt{+} \hlstd{(}\hlnum{1}\hlopt{/}\hlnum{1.06}\hlstd{)}\hlopt{*}\hlnum{0.6} \hlopt{+} \hlstd{(}\hlnum{1}\hlopt{/}\hlnum{1.07}\hlstd{)}\hlopt{*}\hlnum{0.25}
\hlstd{mu_2} \hlkwb{<-} \hlstd{(}\hlnum{1}\hlopt{/}\hlnum{1.05}\hlopt{^}\hlnum{2}\hlstd{)}\hlopt{*}\hlnum{0.15} \hlopt{+} \hlstd{(}\hlnum{1}\hlopt{/}\hlnum{1.06}\hlopt{^}\hlnum{2}\hlstd{)}\hlopt{*}\hlnum{0.6} \hlopt{+} \hlstd{(}\hlnum{1}\hlopt{/}\hlnum{1.07}\hlopt{^}\hlnum{2}\hlstd{)}\hlopt{*}\hlnum{0.25}
\hlstd{mu_3} \hlkwb{<-} \hlstd{(}\hlnum{1}\hlopt{/}\hlnum{1.05}\hlopt{^}\hlnum{3}\hlstd{)}\hlopt{*}\hlnum{0.15} \hlopt{+} \hlstd{(}\hlnum{1}\hlopt{/}\hlnum{1.06}\hlopt{^}\hlnum{3}\hlstd{)}\hlopt{*}\hlnum{0.6} \hlopt{+} \hlstd{(}\hlnum{1}\hlopt{/}\hlnum{1.07}\hlopt{^}\hlnum{3}\hlstd{)}\hlopt{*}\hlnum{0.25}
\end{alltt}
\end{kframe}
\end{knitrout}

Implement recursive function for the first three moments:

$$E\left[a_n\right]=\mu_1\left(1+E\left[a_{n-1}\right]\right)$$
$$E\left[a_n^2\right]=\mu_2\left(1+2E\left[a_{n-1}\right]+E\left[a_{n-1}^2\right]\right)$$
$$E\left[a_n^3\right]=\mu _3\left(1+3E\left[a_{n-1}\right]+3E\left[a_{n-1}^2\right]+E\left[a_{n-1}^3\right]\right)$$


\begin{knitrout}
\definecolor{shadecolor}{rgb}{0.969, 0.969, 0.969}\color{fgcolor}\begin{kframe}
\begin{alltt}
\hlstd{moment_calculator} \hlkwb{<-} \hlkwa{function} \hlstd{(}\hlkwc{n}\hlstd{,} \hlkwc{r}\hlstd{) \{}
  \hlkwa{if} \hlstd{(r} \hlopt{==} \hlnum{1}\hlstd{)}
  \hlstd{\{}
    \hlkwa{if} \hlstd{(n} \hlopt{==} \hlnum{1}\hlstd{) \{}
      \hlkwd{return}\hlstd{( mu_1 )} \hlcom{#Base case}
    \hlstd{\}} \hlkwa{else} \hlstd{\{}
      \hlkwd{return} \hlstd{(mu_1} \hlopt{*} \hlstd{(}\hlnum{1} \hlopt{+} \hlkwd{moment_calculator}\hlstd{(n} \hlopt{-} \hlnum{1}\hlstd{, r)))} \hlcom{#Recursive case}
    \hlstd{\}}
  \hlstd{\}}
  \hlkwa{else if} \hlstd{(r} \hlopt{==} \hlnum{2}\hlstd{)}
  \hlstd{\{}
    \hlkwa{if} \hlstd{(n} \hlopt{==} \hlnum{1}\hlstd{) \{}
      \hlkwd{return}\hlstd{( mu_2 )} \hlcom{#Base case}
    \hlstd{\}} \hlkwa{else} \hlstd{\{}
      \hlkwd{return} \hlstd{(mu_2} \hlopt{*} \hlstd{(}\hlnum{1} \hlopt{+} \hlnum{2} \hlopt{*} \hlkwd{moment_calculator}\hlstd{(n} \hlopt{-} \hlnum{1}\hlstd{,} \hlnum{1}\hlstd{)}
                      \hlopt{+} \hlkwd{moment_calculator}\hlstd{(n} \hlopt{-} \hlnum{1}\hlstd{, r)))} \hlcom{#Recursive case}
    \hlstd{\}}
  \hlstd{\}}
  \hlkwa{else if} \hlstd{(r} \hlopt{==} \hlnum{3}\hlstd{)}
  \hlstd{\{}
    \hlkwa{if} \hlstd{(n} \hlopt{==} \hlnum{1}\hlstd{) \{}
      \hlkwd{return}\hlstd{( mu_3 )} \hlcom{#Base case}
    \hlstd{\}} \hlkwa{else} \hlstd{\{}
      \hlkwd{return} \hlstd{(mu_3} \hlopt{*} \hlstd{(}\hlnum{1} \hlopt{+} \hlnum{3} \hlopt{*} \hlkwd{moment_calculator}\hlstd{(n} \hlopt{-} \hlnum{1}\hlstd{,} \hlnum{1}\hlstd{)} \hlopt{+} \hlnum{3} \hlopt{*} \hlkwd{moment_calculator}\hlstd{(n} \hlopt{-} \hlnum{1}\hlstd{,} \hlnum{2}\hlstd{)}
                      \hlopt{+} \hlkwd{moment_calculator}\hlstd{(n} \hlopt{-} \hlnum{1}\hlstd{, r)))} \hlcom{#Recursive case}
    \hlstd{\}}
  \hlstd{\}}
  \hlkwa{else}
  \hlstd{\{}
    \hlkwd{print}\hlstd{(}\hlstr{"higher moment functionality not yet available."}\hlstd{)}
  \hlstd{\}}
\hlstd{\}}
\end{alltt}
\end{kframe}
\end{knitrout}

\begin{knitrout}
\definecolor{shadecolor}{rgb}{0.969, 0.969, 0.969}\color{fgcolor}\begin{kframe}
\begin{alltt}
\hlcom{#Prepare dataframe for the resultes}
\hlstd{df} \hlkwb{<-} \hlkwd{data.frame}\hlstd{(}\hlkwc{n} \hlstd{=} \hlkwd{c}\hlstd{(}\hlnum{1}\hlopt{:}\hlnum{20}\hlstd{),}
                 \hlkwc{First_Moment} \hlstd{=} \hlkwd{c}\hlstd{(}\hlkwd{rep}\hlstd{(}\hlnum{NA}\hlstd{,} \hlnum{20}\hlstd{)),}
                 \hlkwc{Second_Moment} \hlstd{=} \hlkwd{c}\hlstd{(}\hlkwd{rep}\hlstd{(}\hlnum{NA}\hlstd{,} \hlnum{20}\hlstd{)),}
                 \hlkwc{Third_Moment} \hlstd{=} \hlkwd{c}\hlstd{(}\hlkwd{rep}\hlstd{(}\hlnum{NA}\hlstd{,} \hlnum{20}\hlstd{)))}

\hlcom{#Populate the dataframe}
\hlkwa{for} \hlstd{(r} \hlkwa{in} \hlnum{2}\hlopt{:}\hlnum{4}\hlstd{) \{}
  \hlkwa{for} \hlstd{(n} \hlkwa{in} \hlnum{1}\hlopt{:}\hlnum{20}\hlstd{)\{}
    \hlstd{df[[r]][n]} \hlkwb{=} \hlkwd{moment_calculator}\hlstd{(n, r} \hlopt{-} \hlnum{1}\hlstd{)}
  \hlstd{\}}
\hlstd{\}}

\hlcom{#Table created using Stargazer}
\end{alltt}
\end{kframe}
\end{knitrout}

% Table created by stargazer v.5.2.2 by Marek Hlavac, Harvard University. E-mail: hlavac at fas.harvard.edu
% Date and time: Fri, Aug 21, 2020 - 3:06:16 AM
\begin{table}[!htbp] \centering 
  \caption{Moments} 
  \label{} 
\begin{tabular}{@{\extracolsep{5pt}} cccc} 
\\[-1.8ex]\hline 
\hline \\[-1.8ex] 
  n & First\_Moment & Second\_Moment & Third\_Moment \\ 
\hline \\[-1.8ex] 
 $1$ & $0.943$ & $0.888$ & $0.837$ \\ 
 $2$ & $1.831$ & $3.352$ & $6.139$ \\ 
 $3$ & $2.668$ & $7.120$ & $19.000$ \\ 
 $4$ & $3.457$ & $11.955$ & $41.339$ \\ 
 $5$ & $4.201$ & $17.653$ & $74.175$ \\ 
 $6$ & $4.902$ & $24.036$ & $117.857$ \\ 
 $7$ & $5.563$ & $30.953$ & $172.235$ \\ 
 $8$ & $6.186$ & $38.273$ & $236.810$ \\ 
 $9$ & $6.773$ & $45.882$ & $310.840$ \\ 
 $10$ & $7.327$ & $53.686$ & $393.426$ \\ 
 $11$ & $7.848$ & $61.601$ & $483.580$ \\ 
 $12$ & $8.340$ & $69.561$ & $580.273$ \\ 
 $13$ & $8.803$ & $77.505$ & $682.477$ \\ 
 $14$ & $9.240$ & $85.387$ & $789.188$ \\ 
 $15$ & $9.651$ & $93.164$ & $899.447$ \\ 
 $16$ & $10.039$ & $100.806$ & $1,012.354$ \\ 
 $17$ & $10.405$ & $108.283$ & $1,127.077$ \\ 
 $18$ & $10.750$ & $115.577$ & $1,242.853$ \\ 
 $19$ & $11.075$ & $122.668$ & $1,358.995$ \\ 
 $20$ & $11.381$ & $129.546$ & $1,474.887$ \\ 
\hline \\[-1.8ex]
\end{tabular}
\end{table}
\end{document}
